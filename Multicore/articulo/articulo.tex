%%%
%%%
%%% Plantilla para artículo de Programación multinúcleo
%%%
%%% FAVOR DE NO CAMBIAR FUENTES, MÁRGENES, ETC.
%%%

\documentclass[10pt,letterpaper,oneside]{article}
\usepackage[margin=2.5cm]{geometry}
\usepackage[utf8]{inputenc}
\usepackage{endnotes}
\usepackage{epsfig}

\setlength{\parindent}{0pt}
\setlength{\parskip}{6pt}
\usepackage[spanish,mexico]{babel}

\begin{document}

\renewcommand\abstractname{Resumen}
\renewcommand\refname{Referencias}
\renewcommand{\notesname}{Notas}

\title{Plantilla para artículo de Programación multinúcleo}
\author{
\Large Diego Galíndez Barreda (A01370815)
\\
Tecnológico de Monterrey, Campus Estado de México
\\  
\Large \textit{A01370815@itesm.mx}}  

\maketitle

\begin{abstract}
Este documento, escrito en \LaTeX, sirve como base para la elaboración del artículo de investigación de la materia de \textit{Programación multinúcleo}. 
\end{abstract}

\section{Generalidades}

El trabajo de investigación deberá ser elaborado de manera individual. Cada estudiante seleccionará un problema de programación (distinto a los vistos en clase o encargados de tarea) que pueda ser paralelizado usando al menos tres de las siete herramientas que fueron cubiertas en clase:
    
\begin{itemize}
    \item \textit{Threads} en Java 
    \item Fork/Join \textit{framework} en Java
    \item \textit{Streams} paralelos en Java 8
    \item \textit{Web Workers} en JavaScript
    \item Procesos en Erlang    
    \item OpenMP en C       
    \item Intrínsicos SIMD en C
\end{itemize}

El problema seleccionado también debe ser resuelto de manera secuencial en cada uno de los lenguajes utilizados. 

Se deben comparar los tiempos de ejecución de los programas escritos en los diferentes lenguajes y herramientas, tanto en sus versiones secuenciales como paralelas. Así mismo, se deben calcular los diferentes \textit{speedups} obtenidos y se deben realizar una o varias gráficas comparativas que presenten la información de manera clara.   

Los resultados de la investigación deberán reportarse como un artículo de divulgación escrita en idioma español, utilizando el formato de este documento en {\LaTeX}\endnote{ El aprendizaje de {\LaTeX} corre a cargo de cada estudiante.}. El artículo deberá tener una extensión de por lo menos seis páginas (sin incluir los apéndices), y debe cumplir con los lineamientos de los elementos formales descritos en la \textit{Guía de presentación de trabajos escritos}~\cite{guia}. En particular, el trabajo debe contar con las siguientes secciones:
    
\begin{itemize}
    \item \textbf{Datos:} título y autor de la investigación. 
    \item \textbf{Resumen:} contenido abreviado y preciso del artículo. 
    \item \textbf{Introducción:} descripción del problema que se va a resolver.
    \item \textbf{Desarrollo:} explicación de la manera en que se resolvió el problema usando cada una de las herramientas.
    \item \textbf{Conclusiones:} Enunciar las conclusiones personales sobre los resultados obtenidos.
    \item \textbf{Agradecimientos:} esta sección es opcional.
    \item \textbf{Referencias:} libros, revistas, sitios web, etc. que fueron consultados.
    \item \textbf{Apéndices:} incluir en esta sección todos los códigos fuentes que fueron elaborados para la investigación.
\end{itemize}

El público al que está dirigido el artículo son los mismos compañeros de esta clase, así que se debe tener esto en mente al momento de realizar la redacción.

Se evaluará la calidad técnica del artículo, así como su redacción, ortografía y la claridad y orden de la presentación de las ideas. 

La fecha de entrega de este trabajo es el día del examen final (1 de diciembre del 2015).

\section{Consejos adicionales}

\subsection{Instalando {\LaTeX} en Ubuntu}

Para instalar {\LaTeX} de manera completa en Ubuntu solo se debe teclear la siguiente instrucción desde la terminal:

\begin{verbatim}
    sudo apt-get install texlive-full
\end{verbatim}

Esta instrucción toma algo de tiempo ya que requiere descargar más de 900 MB. 

\subsection{Código fuente}

En {\LaTeX} se debe usar el ambiente \verb!verbatim! para colocar listados de código fuente. Por ejemplo:  

\begin{verbatim}
    #include <stdio.h>
    
    int main() {
        printf("Hola Mundo!\n");
        return 0;
    }
\end{verbatim}

Los elementos de un programa que aparecen en un párrafo de texto normal (como el nombre de la función \verb!main!) deben ir dentro del comando \verb!\verb! o \verb!\texttt!.

\subsection{Fórmulas}

{\LaTeX} fue hecho pensando en el manejo de fórmulas matemáticas, por ejemplo:

\begin{displaymath}
\overline{x} = \frac{1}{n}\sum_{i=1}^n x_{i}
\end{displaymath}

\subsection{Tablas}

Así se ven las tablas en {\LaTeX}:

\begin{center}

    \begin{tabular}{|r|r|}
    \hline
    \hline
    $n$ & $n!$  \\
    \hline
    \hline
    0 & 1         \\
    1 & 1         \\
    2 & 2         \\
    3 & 6         \\
    4 & 24        \\
    5 & 120       \\
    6 & 720       \\
    7 & 5,040      \\
    8 & 40,320     \\
    9 & 362,880    \\    
    \hline
    \end{tabular}
    
\end{center}    
    
\subsection{Imágenes}

Las imágenes que se deseen agregar al documento deben ser de alta calidad y deben estar en formato EPS\endnote{ LibreOffice Draw permite crear este tipo de imágenes con relativa facilidad.}. A continuación se presenta un ejemplo: 

\centerline{\epsfig{figure=dibujo.eps,width=3in}}
   
\subsection{Más información sobre \LaTeX}

La referencia completa y manual de usuario de {\LaTeX} está disponible en~\cite{lamport}. Y recuerda, si necesitas más información, ¡Google es tu mejor amigo!~\cite{giybf}.

\section{Conclusiones}

Lleva algo de tiempo aprender a usar {\LaTeX} correctamente. Sin embargo vale la pena por el resultado que se obtiene: documentos bellamente formateados. 

\section{Agradecimientos}

En esta sección se puede agradecer a las personas que ayudaron de cualquier manera a elaborar el trabajo. 

\theendnotes
 
\begin{thebibliography}{9}
    
\bibitem{guia}  
    Departamento de letras del ITESM CEM. 
    \emph{Guía de presentación de trabajos escritos.} \\
    http://www.cem.itesm.mx/consulta/guia/ Accedido el 20 de octubre del 2015.
    
\bibitem{giybf}
    GIYBF. 
    \emph{Google is your best Friend!} \\
    http://www.giybf.com/ Accedido el 20 de octubre del 2015.
    
\bibitem{lamport}
    Lamport, L. 
    \emph{\LaTeX: A Document Preparation System, 2nd Edition.} 
    Addison-Wesley Professional, 1994.

\end{thebibliography}

\end{document}
